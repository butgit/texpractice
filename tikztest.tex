\documentclass{article}
\usepackage{tikz}
\usepackage[active,tightpage]{preview}
\PreviewEnvironment{tikzpicture}
\setlength\PreviewBorder{5pt}%
\usetikzlibrary{fit}
\usetikzlibrary{shapes.geometric}

\begin{document}
\begin{tikzpicture}[scale=1,cap=round,>=latex]
\node[draw,fill=blue,rectangle] (v1) at (0,0) {$v_1$} ;
\node[draw,fill=red,circle] (v2) at (2,0) {$v_2$} ;
\draw[<-] (v1) to (v2) ;
\node[draw,ellipse,fit=(v1) (v2)] (ell) {} ;
\end{tikzpicture}
\begin{tikzpicture}[scale=1,cap=round,>=latex]
\node[draw,fill=blue,rectangle] (v1) at (0,0) {$v_1$} ;
\node[draw,fill=red,circle] (v2) at (2,0) {$v_2$} ;
\draw[<-] (v1) to (v2) ;
\node[draw,rectangle,rounded corners,very thick,
fit=(v1) (v2)] (rec) {} ;
\end{tikzpicture}
\begin{tikzpicture}[scale=1,cap=round,>=latex]
\foreach \i in {1,3,...,13}{
    \begin{scope}[xshift=\i cm, rotate=15*\i]
    \fill[red,opacity=\i / 13] (0,0) -- (1,0) -- (0.5,0.7);
    \end{scope}
}
\end{tikzpicture}
\begin{tikzpicture}[scale=1,cap=round,>=latex]
\foreach \i in {1,...,10}{
    \foreach \j in {1,...,10}{
        \pgfmathtruncatemacro{\ij}{\i * \j}
        \node (\i-\j) at (\i,\j) {$\ij$} ;}}
\fill[opacity=0.3,purple] (9-4) circle (0.6) ;
\fill[opacity=0.3,cyan] (6-6) circle (0.6) ;
\draw[<->,very thick] (9-4) -- (6-6) ;
\end{tikzpicture}
\end{document}